\documentclass{article}

\usepackage{fancyhdr}
\usepackage{extramarks}
\usepackage{amsmath}
\usepackage{amsthm}
\usepackage{amsfonts}
\usepackage{tikz}
\usepackage[plain]{algorithm}
\usepackage{algpseudocode}
\usepackage{enumerate}
\usepackage{tikz}
\usepackage{pgfplots}
\usepackage{graphicx}
\usepackage{ctex}
\usepackage{xcolor}

\usepgfplotslibrary{fillbetween} 

\usetikzlibrary{automata,positioning}

%
% Basic Document Settings
%  

\topmargin=-0.45in
\evensidemargin=0in
\oddsidemargin=0in
\textwidth=6.5in
\textheight=9.0in
\headsep=0.25in

\linespread{1.1}

\pagestyle{fancy}
\lhead{\hmwkAuthorName}
\chead{\hmwkClass : \hmwkTitle}
\rhead{\firstxmark}
\lfoot{\lastxmark}
\cfoot{\thepage}

\renewcommand\headrulewidth{0.4pt}
\renewcommand\footrulewidth{0.4pt}

\setlength\parindent{0pt}

%
% Create Problem Sections
%

\newcommand{\enterProblemHeader}[1]{
    \nobreak\extramarks{}{Problem \arabic{#1} continued on next page\ldots}\nobreak{}
    \nobreak\extramarks{Problem \arabic{#1} (continued)}{Problem \arabic{#1} continued on next page\ldots}\nobreak{}
}

\newcommand{\exitProblemHeader}[1]{
    \nobreak\extramarks{Problem \arabic{#1} (continued)}{Problem \arabic{#1} continued on next page\ldots}\nobreak{}
    \stepcounter{#1}
    \nobreak\extramarks{Problem \arabic{#1}}{}\nobreak{}
}

\newcommand*\circled[1]{\tikz[baseline=(char.base)]{
		\node[shape=circle,draw,inner sep=2pt] (char) {#1};}}


\setcounter{secnumdepth}{0}
\newcounter{partCounter}
\newcounter{homeworkProblemCounter}
\setcounter{homeworkProblemCounter}{1}
\nobreak\extramarks{Problem \arabic{homeworkProblemCounter}}{}\nobreak{}

%
% Homework Problem Environment
%
% This environment takes an optional argument. When given, it will adjust the
% problem counter. This is useful for when the problems given for your
% assignment aren't sequential. See the last 3 problems of this template for an
% example.
%

\newenvironment{homeworkProblem}[1][-1]{
    \ifnum#1>0
        \setcounter{homeworkProblemCounter}{#1}
    \fi
    \section{Problem \arabic{homeworkProblemCounter}}
    \setcounter{partCounter}{1}
    \enterProblemHeader{homeworkProblemCounter}
}{
    \exitProblemHeader{homeworkProblemCounter}
}

%
% Homework Details
%   - Title
%   - Class
%   - Due date
%   - Name
%   - Student ID

\newcommand{\hmwkTitle}{Lab Report}
\newcommand{\hmwkClass}{General Physics}
\newcommand{\hmwkDueDate}{Nov 19, 2023}
\newcommand{\hmwkAuthorName}{Runkang Yang}
\newcommand{\hmwkAuthorID}{2022533080}


%
% Title Page
%

\title{
    \vspace{2in}
    \textmd{\textbf{\hmwkClass:\\  \hmwkTitle}}\\
    \normalsize\vspace{0.1in}\small{Due\ on\ \hmwkDueDate\ at 23:59}\\
	\vspace{4in}
}

\author{
	Name: \textbf{\hmwkAuthorName} \\
	Student ID: \hmwkAuthorID}
\date{}

\renewcommand{\part}[1]{\textbf{\large Part \Alph{partCounter}}\stepcounter{partCounter}\\}

%
% Various Helper Commands
%

% Useful for algorithms
\newcommand{\alg}[1]{\textsc{\bfseries \footnotesize #1}}
% For derivatives
\newcommand{\deriv}[1]{\frac{\mathrm{d}}{\mathrm{d}x} (#1)}
% For partial derivatives
\newcommand{\pderiv}[2]{\frac{\partial}{\partial #1} (#2)}
% Integral dx
\newcommand{\dx}{\mathrm{d}x}
% Alias for the Solution section header
\newcommand{\solution}{\textbf{\large Solution}}
% Probability commands: Expectation, Variance, Covariance, Bias
\newcommand{\E}{\mathrm{E}}
\newcommand{\Var}{\mathrm{Var}}
\newcommand{\Cov}{\mathrm{Cov}}
\newcommand{\Bias}{\mathrm{Bias}}

\begin{document}

\maketitle

\pagebreak
%====================================================
\begin{homeworkProblem}[1]

\begin{enumerate}
    % inline
    \item 
{\Large \textbf{\textcolor{red}{计算公式}}}

\vspace{\baselineskip}

1.算术平均值\quad 
\textcolor{purple}{
$$\bar{N} =\frac{\sum_{i=1}^nN_i}{n}$$
}
2.测量值的标准偏差\quad
\textcolor{purple}{
$$\sigma_N=\sqrt{\frac{\sum_{i=1}^n(N_i-\bar{N})^2}{n-1}}$$
}
3.平均值的标准偏差\quad
\textcolor{purple}{
$$\sigma_{\bar{N}}=\frac{\sigma_N}{\sqrt{n}}=\sqrt{\frac{\sum_{i=1}^n(N_i-\bar{N})^2}{n(n-1)}}$$
}
4.粗差的剔除\quad 拉依达准则 \quad 
\textcolor{purple}{
$$\left| \Delta N_i \right|<3\sigma_N$$
}
5.不确定度\quad 
\textcolor{purple}{
$$U=\sqrt{\Delta_A^2+\Delta_B^2}$$
}

\vspace{\baselineskip}
\textbf{\textcolor{blue}{Step1:计算根据测得的小球直径计算出体积,进而利用$\rho=\frac{M}{V}$求出小球密度}}

\textbf{\textcolor{yellow!80!black}{\large 1号小球}}\\
小球质量:3.19$g$\\
小球体积: [0.604, 0.605, 0.605, 0.605, 0.605, 0.605]\\
小球密度: [5.279, 5.276, 5.273, 5.275, 5.276, 5.273]\\
\begin{align*}  % 可以使用 align 代替 align* 来显示行号
\bar{N}&=\frac{1}{6}(5.279+5.276+5.273+5.275+5.276+5.273)=5.275 \\
\sigma_N&=0.0023 \quad \sigma_{\bar{N}}=0.00092 \quad U=\frac{t_0.95}{\sqrt{n}}\sigma_N=0.0024
\end{align*}
利用定义求得1号小球的密度的平均测量值为
\textcolor{green!70!black}{
$$
\rho_1=(\rho \pm U)=(5.275\pm 0.0024) \, \text{g/cm}^3
$$
}
\textbf{\textcolor{yellow!80!black}{\large 2号小球}}\\
小球质量:1.05$g$\\
小球体积: [0.261, 0.261, 0.261, 0.261, 0.261, 0.26]\\
小球密度: [4.024, 4.03, 4.027, 4.029, 4.027, 4.032]\\
\begin{align*}  % 可以使用 align 代替 align* 来显示行号
\bar{N}&=\frac{1}{6}(4.024+4.03+4.027+4.029+4.027+4.032)=4.028 \\
\sigma_N&=0.0027 \quad \sigma_{\bar{N}}=0.0011 \quad U=\frac{t_0.95}{\sqrt{n}}\sigma_N=0.0029
\end{align*}
利用定义求得2号小球的密度的平均测量值为
\textcolor{green!70!black}{
$$
\rho_2=(\rho \pm U)=(4.028\pm 0.0029)g/cm^3
$$}
\textbf{\textcolor{yellow!80!black}{\large 3号小球}}\\
小球质量:0.88$g$\\
小球体积: [0.261, 0.262, 0.261, 0.261, 0.261, 0.261]\\
小球密度: [3.369, 3.364, 3.368, 3.369, 3.37, 3.369]\\
\begin{align*}  % 可以使用 align 代替 align* 来显示行号
\bar{N}&=\frac{1}{6}(3.369+3.364+3.368+3.369+3.37+3.369)=3.368 \\
\sigma_N&=0.0021 \quad \sigma_{\bar{N}}=0.00087 \quad U=\frac{t_0.95}{\sqrt{n}}\sigma_N=0.0022
\end{align*}
利用定义求得3号小球的密度的平均测量值为
\textcolor{green!70!black}{
$$
\rho_3=(\rho \pm U)=(3.368\pm 0.0022)g/cm^3
$$}

\vspace{\baselineskip}
\textbf{\textcolor{blue}{Step2:利用流体静力称衡法得出的小球的密度}}

\textbf{\textcolor{yellow!80!black}{\large 1号小球}}\\
六组数据密度的测量值分别为\quad [7.25, 4.833, 5.596, 5.907, 5.596, 5.23]
\begin{align*}  % 可以使用 align 代替 align* 来显示行号
\bar{N}&=\frac{1}{6}(7.25+4.833+5.596+5.907+5.596+5.23)=5.735 \\
\sigma_N&=0.828 \quad \sigma_{\bar{N}}=0.338 \quad U=\frac{t_0.95}{\sqrt{n}}\sigma_N=0.870
\end{align*}
求得1号小球的密度的平均测量值为
\textcolor{green!70!black}{
$$
\rho_1=(\rho \pm U)=(5.735\pm 0.870)g/cm^3
$$}

\textbf{\textcolor{yellow!80!black}{\large 2号小球}}\\
六组数据密度的测量值分别为\quad [4.038, 4.2, 4.2, 3.889, 4.038, 4.038]
\begin{align*}  % 可以使用 align 代替 align* 来显示行号
\bar{N}&=\frac{1}{6}(4.038+4.2+4.2+3.889+4.038+4.038)=4.067 \\
\sigma_N&=0.118 \quad \sigma_{\bar{N}}=0.048 \quad U=\frac{t_0.95}{\sqrt{n}}\sigma_N=0.124
\end{align*}
求得2号小球的密度的平均测量值为
\textcolor{green!70!black}{
$$
\rho_1=(\rho \pm U)=(4.067\pm 0.124)g/cm^3
$$}
\textbf{\textcolor{yellow!80!black}{\large 3号小球}}\\
六组数据密度的测量值分别为\quad [3.52, 3.52, 3.385, 3.385, 3.52, 3.385]
\begin{align*}  % 可以使用 align 代替 align* 来显示行号
\bar{N}&=\frac{1}{6}(3.52+3.52+3.385+3.385+3.52+3.385)=3.453 \\
\sigma_N&=0.074 \quad \sigma_{\bar{N}}=0.030 \quad U=\frac{t_0.95}{\sqrt{n}}\sigma_N=0.078
\end{align*}
求得3号小球的密度的平均测量值为
\textcolor{green!70!black}{
$$
\rho_1=(\rho \pm U)=(3.453\pm 0.078)g/cm^3
$$}
$$
\frac{2^{h+1}-1}{2^h}=2 \times \frac{2^h}{2^h}-\frac{1}{2^h}=2-0=2
$$












\end{enumerate}

\end{homeworkProblem}
%==========================================================换页==============================
\clearpage
\begin{homeworkProblem}[2]

\begin{enumerate}
	% alignment
	\item

{\Large \textbf{\textcolor{red}{计算公式}}}

\vspace{\baselineskip}

1.算术平均值\quad 
\textcolor{purple}{
$$\bar{N} =\frac{\sum_{i=1}^nN_i}{n}$$
}
2.测量值的标准偏差\quad
\textcolor{purple}{
$$\sigma_N=\sqrt{\frac{\sum_{i=1}^n(N_i-\bar{N})^2}{n-1}}$$
}
3.平均值的标准偏差\quad
\textcolor{purple}{
$$\sigma_{\bar{N}}=\frac{\sigma_N}{\sqrt{n}}=\sqrt{\frac{\sum_{i=1}^n(N_i-\bar{N})^2}{n(n-1)}}$$
}
4.粗差的剔除\quad 拉依达准则 \quad 
\textcolor{purple}{
$$\left| \Delta N_i \right|<3\sigma_N$$
}
5.不确定度\quad 
\textcolor{purple}{
$$U=\sqrt{\Delta_A^2+\Delta_B^2}$$
}

\vspace{\baselineskip}
\large \textbf{\textcolor{blue}{Part1:用新型焦利秤测定弹簧的劲度系数K}}\\

\textcolor{green!70!black}{①通过添加砝码作出的F-Δy图像如下}\\



斜率$$K=0.1823g/mm$$结合g=9.794N/kg测得的弹簧的劲度系数为$$K=1.785N/m$$
\textcolor{green!70!black}{②通过减少砝码作出的F-Δy图像如下}


斜率$$K=0.1796g/mm$$结合g=9.794N/kg测得的弹簧的劲度系数为$$K=1.759N/m$$


\large \textbf{\textcolor{blue}{Part2:测量弹簧简谐振动周期,计算得出弹簧的劲度系数K}}

铁砝码质量:[21.50,21.48,21.48,21.47,21.46]
\begin{align*}  % 可以使用 align 代替 align* 来显示行号
\bar{N}&=\frac{1}{5}(21.50+21.48+21.48+21.47+21.46)=21.48 \\
\sigma_N&=0.0148 \quad \sigma_{\bar{N}}=0.066 \quad U=\frac{t_0.95}{\sqrt{n}}\sigma_N=0.02
\end{align*}
\quad 求得铁砝码质量的平均测量值为
\textcolor{green!70!black}{
$$
M=(M \pm U)=(21.48\pm 0.02)g
$$}
弹簧质量:[13.60,13.58,13.54,13.54,13.56]
\begin{align*}  % 可以使用 align 代替 align* 来显示行号
\bar{N}&=\frac{1}{5}(13.60+13.58+13.54+13.54+13.56)=13.56 \\
\sigma_N&=0.0260 \quad \sigma_{\bar{N}}=0.012 \quad U=\frac{t_0.95}{\sqrt{n}}\sigma_N=0.03
\end{align*}
\quad 求得弹簧质量的平均测量值为
\textcolor{green!70!black}{
$$
M_0=(M_0 \pm U)=(13.56\pm 0.03)g
$$}
10T:[7.592,7.586,7.602,7.594,7.500,7.590,7.590,7.586,7.592,7.593]\\
T:[0.7592,0.7586,0.7602,0.7594,0.750,0.759,0.759,0.7586,0.7592,0.7593]
\begin{align*}  % 可以使用 align 代替 align* 来显示行号
\bar{N}&=\frac{1}{10}(0.7592+0.7586+0.7602+0.7594+0.750+0.759+0.759+0.7586+0.7592+0.7593)\\
&=0.758 \\
\sigma_N&=0.0029 \quad \sigma_{\bar{N}}=0.00092 \quad U=\frac{t_0.95}{\sqrt{n}}\sigma_N=0.010
\end{align*}
\quad 求得T的平均测量值为
\textcolor{green!70!black}{
$$
T=(T \pm U)=(0.758\pm 0.010)s
$$}
结合公式$$T=2\pi \sqrt{\frac{M+PM_0}{K}}$$可得$$K=\frac{(4\pi^2)(M+PM_0)}{T^2}$$将$$p=\frac{1}{3}$$带入并进行单位转换可得$$K=1.786N/m$$












\end{enumerate}

\end{homeworkProblem}

%==========================================================换页==============================
\clearpage
\begin{homeworkProblem}[3]

\begin{enumerate}
	% alignment
	\item

{\Large \textbf{\textcolor{red}{计算公式}}}

\vspace{\baselineskip}

1.算术平均值\quad 
\textcolor{purple}{
$$\bar{N} =\frac{\sum_{i=1}^nN_i}{n}$$
}
2.测量值的标准偏差\quad
\textcolor{purple}{
$$\sigma_N=\sqrt{\frac{\sum_{i=1}^n(N_i-\bar{N})^2}{n-1}}$$
}
3.平均值的标准偏差\quad
\textcolor{purple}{
$$\sigma_{\bar{N}}=\frac{\sigma_N}{\sqrt{n}}=\sqrt{\frac{\sum_{i=1}^n(N_i-\bar{N})^2}{n(n-1)}}$$
}
4.粗差的剔除\quad 拉依达准则 \quad 
\textcolor{purple}{
$$\left| \Delta N_i \right|<3\sigma_N$$
}
5.不确定度\quad 
\textcolor{purple}{
$$U=\sqrt{\Delta_A^2+\Delta_B^2}$$
}

\vspace{\baselineskip}
\large \textbf{\textcolor{blue}{Part1:小球直径d}}\\
小球直径:[1.030,1.033,1.029,1.032,1.030]\quad 单位:mm
$$
\bar{N}=\frac{1}{5}(1.030+1.033+1.029+1.032+1.030)=1.031
$$
$$
\sigma_N=0.0016 \quad \sigma_{\bar{N}}=0.00073 \quad \Delta A=0.0019 \quad \Delta B=0.0095 \quad U=\sqrt{\Delta A^2+\Delta B^2}=0.010
$$
\quad 求得小球直径的平均测量值为
\textcolor{green!70!black}{
$$
d=(d \pm U)=(1.031\pm 0.010)mm
$$}
%\textcolor{green!70!black}{①通过添加砝码作出的F-Δy图像如下}\\
\large \textbf{\textcolor{blue}{Part2:v0的测定}}\\
\textcolor{blue!50!white}{19°C \quad t:[29.15,29.12,29.12,29.26,29.02]}
$$
\bar{N}=\frac{1}{5}(29.15+29.12+29.12+29.26+29.02)=29.13
$$
$$
\sigma_N=0.086 \quad \sigma_{\bar{N}}=0.038 \quad \Delta A=0.0987 \quad \Delta B=0.0095 \quad U=\sqrt{\Delta A^2+\Delta B^2}=0.10
$$
\quad 求得时间的平均测量值为
\textcolor{green!60!black}{
$$
t=(t \pm U)=(29.13\pm 0.10)s
$$}
速度 \textcolor{green!30!black}{
$$
v_0=\frac{L}{\bar{t}}=\frac{0.1}{29.13}=3.43\times 10^{-3} m/s
$$}
粘度$\eta$ \textcolor{red!30!blue}{
$$
\eta=\frac{(\rho-\rho_0)gd^2}{18v_0(1+2.4\frac{d}{D})}=1.026 Pa\cdot s
$$}
修正粘度$\eta_1$ \textcolor{red!30!blue}{
$$
\eta_1=\eta-\frac{3}{16}v_0d\rho_0=1.025 Pa\cdot s
$$}

\textcolor{blue!50!white}{21°C \quad t:[25.44,25.19,25.21,25.67,25.28]}
$$
\bar{N}=\frac{1}{5}(25.44+25.19+25.2+25.67+25.28)=25.36
$$
$$
\sigma_N=0.200 \quad \sigma_{\bar{N}}=0.089 \quad \Delta A=0.23 \quad \Delta B=0.01 \quad U=\sqrt{\Delta A^2+\Delta B^2}=0.23
$$
\quad 求得时间的平均测量值为
\textcolor{green!60!black}{
$$
t=(t \pm U)=(25.36\pm 0.23)s
$$}
速度 \textcolor{green!30!black}{
$$
v_0=\frac{L}{\bar{t}}=\frac{0.1}{25.36}=3.94\times 10^{-3} m/s
$$}
粘度$\eta$ \textcolor{red!30!blue}{
$$
\eta=\frac{(\rho-\rho_0)gd^2}{18v_0(1+2.4\frac{d}{D})}=0.893 Pa\cdot s
$$}
修正粘度$\eta_1$ \textcolor{red!30!blue}{
$$
\eta_1=\eta-\frac{3}{16}v_0d\rho_0=0.892 Pa\cdot s
$$}

\textcolor{blue!50!white}{23°C \quad t:[22.70,22.55,22.17,22.01,21.95]}
$$
\bar{N}=\frac{1}{5}(22.70+22.55+22.17+22.01+21.95)=22.28
$$
$$
\sigma_N=0.332 \quad \sigma_{\bar{N}}=0.149 \quad \Delta A=0.38 \quad \Delta B=0.01 \quad U=\sqrt{\Delta A^2+\Delta B^2}=0.38
$$
\quad 求得时间的平均测量值为
\textcolor{green!60!black}{
$$
t=(t \pm U)=(22.28\pm 0.38)s
$$}
速度 \textcolor{green!30!black}{
$$
v_0=\frac{L}{\bar{t}}=\frac{0.1}{22.28}=4.49\times 10^{-3} m/s
$$}
粘度$\eta$ \textcolor{red!30!blue}{
$$
\eta=\frac{(\rho-\rho_0)gd^2}{18v_0(1+2.4\frac{d}{D})}=0.783 Pa\cdot s
$$}
修正粘度$\eta_1$ \textcolor{red!30!blue}{
$$
\eta_1=\eta-\frac{3}{16}v_0d\rho_0=0.782 Pa\cdot s
$$}

\textcolor{blue!50!white}{25°C \quad t:[20.09,19.55,19.18,18.93,18.87]}
$$
\bar{N}=\frac{1}{5}(20.09+19.55+19.18+18.93+18.87)=19.32
$$
$$
\sigma_N=0.505 \quad \sigma_{\bar{N}}=0.226 \quad \Delta A=0.58 \quad \Delta B=0.01 \quad U=\sqrt{\Delta A^2+\Delta B^2}=0.58
$$
\quad 求得时间的平均测量值为
\textcolor{green!60!black}{
$$
t=(t \pm U)=(19.32\pm 0.58)s
$$}
速度 \textcolor{green!30!black}{
$$
v_0=\frac{L}{\bar{t}}=\frac{0.1}{19.32}=5.12\times 10^{-3} m/s
$$}
粘度$\eta$ \textcolor{red!30!blue}{
$$
\eta=\frac{(\rho-\rho_0)gd^2}{18v_0(1+2.4\frac{d}{D})}=0.687 Pa\cdot s
$$}
修正粘度$\eta_1$ \textcolor{red!30!blue}{
$$
\eta_1=\eta-\frac{3}{16}v_0d\rho_0=0.686 Pa\cdot s
$$}

\textcolor{blue!50!white}{27°C \quad t:[17.08,16.66,17.05,16.60,16.33]}
$$
\bar{N}=\frac{1}{5}(17.08+16.66+17.05+16.60+16.33)=16.74
$$
$$
\sigma_N=0.318 \quad \sigma_{\bar{N}}=0.142 \quad \Delta A=0.36 \quad \Delta B=0.01 \quad U=\sqrt{\Delta A^2+\Delta B^2}=0.36
$$
\quad 求得时间的平均测量值为
\textcolor{green!60!black}{
$$
t=(t \pm U)=(16.74\pm 0.36)s
$$}
速度 \textcolor{green!30!black}{
$$
v_0=\frac{L}{\bar{t}}=\frac{0.1}{16.74}=5.97\times 10^{-3} m/s
$$}
粘度$\eta$ \textcolor{red!30!blue}{
$$
\eta=\frac{(\rho-\rho_0)gd^2}{18v_0(1+2.4\frac{d}{D})}=0.589 Pa\cdot s
$$}
修正粘度$\eta_1$ \textcolor{red!30!blue}{
$$
\eta_1=\eta-\frac{3}{16}v_0d\rho_0=0.588 Pa\cdot s
$$}

\textcolor{blue!50!white}{29°C \quad t:[15.03,14.96,14.44,14.09,14.08]}
$$
\bar{N}=\frac{1}{5}(15.03+14.96+14.44+14.09+14.08)=14.52
$$
$$
\sigma_N=0.458 \quad \sigma_{\bar{N}}=0.205 \quad \Delta A=0.52 \quad \Delta B=0.01 \quad U=\sqrt{\Delta A^2+\Delta B^2}=0.52
$$
\quad 求得时间的平均测量值为
\textcolor{green!60!black}{
$$
t=(t \pm U)=(14.52\pm 0.52)s
$$}
速度 \textcolor{green!30!black}{
$$
v_0=\frac{L}{\bar{t}}=\frac{0.1}{14.52}=6.88\times 10^{-3} m/s
$$}
粘度$\eta$ \textcolor{red!30!blue}{
$$
\eta=\frac{(\rho-\rho_0)gd^2}{18v_0(1+2.4\frac{d}{D})}=0.511 Pa\cdot s
$$}
修正粘度$\eta_1$ \textcolor{red!30!blue}{
$$
\eta_1=\eta-\frac{3}{16}v_0d\rho_0=0.510 Pa\cdot s
$$}


\large \textbf{\textcolor{blue}{Part3:图像拟合}}\\
\textcolor{green!70!black}{η-T表格,图像如下}

\textcolor{red!30!blue}{
分析可知蓖麻油粘度随温度的升高而降低,且近似成指数关系(拟合方程已在图中标出)
}










\end{enumerate}

\end{homeworkProblem}


%==========================================================换页==============================
\clearpage
\begin{homeworkProblem}[4]

\begin{enumerate}
	% alignment
	\item


{\Large \textbf{\textcolor{red}{计算公式}}}

\vspace{\baselineskip}

1.算术平均值\quad 
\textcolor{purple}{
$$\bar{N} =\frac{\sum_{i=1}^nN_i}{n}$$
}
2.测量值的标准偏差\quad
\textcolor{purple}{
$$\sigma_N=\sqrt{\frac{\sum_{i=1}^n(N_i-\bar{N})^2}{n-1}}$$
}
3.平均值的标准偏差\quad
\textcolor{purple}{
$$\sigma_{\bar{N}}=\frac{\sigma_N}{\sqrt{n}}=\sqrt{\frac{\sum_{i=1}^n(N_i-\bar{N})^2}{n(n-1)}}$$
}
4.粗差的剔除\quad 拉依达准则 \quad 
\textcolor{purple}{
$$\left| \Delta N_i \right|<3\sigma_N$$
}
5.不确定度\quad 
\textcolor{purple}{
$$U=\sqrt{\Delta_A^2+\Delta_B^2}$$
}

\vspace{\baselineskip}
\large \textbf{\textcolor{blue}{Part1:圆环外径D1与D2的测定}}\\
$D_1$:[35.04,35.06,35.06,35.02,35.04]\quad 单位:$\times 10^{-3}m$
$$
\bar{N}=\frac{1}{5}(35.04+35.06+35.06+35.02+35.04)=35.04
$$
$$
\sigma_N=0.017 \quad \sigma_{\bar{N}}=0.007 \quad \Delta A=0.019 \quad \Delta B=0.0095 \quad U=\sqrt{\Delta A^2+\Delta B^2}=0.02
$$
\quad 求得圆环外径$D_1$的平均测量值为
\textcolor{green!70!black}{
$$
D_1=(D_1 \pm U)=(35.04\pm 0.02) \times 10^{-3}m
$$}

$D_2$:[33.02,33.02,33.00,33.02,33.00]\quad 单位:$\times 10^{-3}m$
$$
\bar{N}=\frac{1}{5}(33.02+33.02+33.00+33.02+33.00)=33.01
$$
$$
\sigma_N=0.011 \quad \sigma_{\bar{N}}=0.005 \quad \Delta A=0.013 \quad \Delta B=0.0095 \quad U=\sqrt{\Delta A^2+\Delta B^2}=0.02
$$
\quad 求得圆环内径$D_2$的平均测量值为
\textcolor{green!70!black}{
$$
D_2=(D_2 \pm U)=(33.01\pm 0.02) \times 10^{-3}m
$$}
%\textcolor{green!70!black}{①通过添加砝码作出的F-Δy图像如下}\\


\large \textbf{\textcolor{blue}{Part2:图像拟合 U=Bf}}\\
\textcolor{green!70!black}{U-f拟合直线,图像如下}


%\textcolor{red!30!blue}{
%B=3.00
%}




\large \textbf{\textcolor{blue}{Part3:求解水的表面张力系数}}\\
\textcolor{blue!50!white}{ΔU \quad ΔU:[39.8,39.5,40.6,38.2,40.7,41.0]} 单位:$\times 10^{-3} V$
$$
\bar{N}=\frac{1}{6}(39.8+39.5+40.6+38.2+40.7+41.0)=40.0
$$
$$
\sigma_N=1.036 \quad \sigma_{\bar{N}}=0.423 \quad \Delta A=1.088 \quad \Delta B=0.0095 \quad U=\sqrt{\Delta A^2+\Delta B^2}=1.1
$$
\quad 求得ΔU的平均测量值为
\textcolor{green!60!black}{
$$
\Delta U=(\Delta U\pm U)=(40.0\pm 1.1) \times 10^{-3}V
$$}
表面张力 \textcolor{green!30!black}{
$$
f=(U_1-U_2)/B=\frac{\Delta U}{B}=13.3\times 10^{-3}N
$$}
表面张力系数$\alpha$ \textcolor{red!30!blue}{
$$
\alpha=\frac{f}{\pi(D_1+D_2)}=0.062N/m
$$}











\end{enumerate}

\end{homeworkProblem}

%==========================================================换页==============================
\clearpage
\begin{homeworkProblem}[5]

\begin{enumerate}
	% alignment
	\item

{\Large \textbf{\textcolor{red}{计算公式}}}

\vspace{\baselineskip}

1.算术平均值\quad 
\textcolor{purple}{
$$\bar{N} =\frac{\sum_{i=1}^nN_i}{n}$$
}
2.测量值的标准偏差\quad
\textcolor{purple}{
$$\sigma_N=\sqrt{\frac{\sum_{i=1}^n(N_i-\bar{N})^2}{n-1}}$$
}
3.平均值的标准偏差\quad
\textcolor{purple}{
$$\sigma_{\bar{N}}=\frac{\sigma_N}{\sqrt{n}}=\sqrt{\frac{\sum_{i=1}^n(N_i-\bar{N})^2}{n(n-1)}}$$
}
4.粗差的剔除\quad 拉依达准则 \quad 
\textcolor{purple}{
$$\left| \Delta N_i \right|<3\sigma_N$$
}
5.不确定度\quad 
\textcolor{purple}{
$$U=\sqrt{\Delta_A^2+\Delta_B^2}$$
}

\vspace{\baselineskip}
\large \textbf{\textcolor{blue}{Part1:已知量的测量}}\\
\textcolor{blue!50!white}{砝码质量\quad$m$:[39.2,39.2,39.2,39.1,39.2,39.2]\quad 单位:$\times 10^{-3}kg$}
$$
\bar{N}=\frac{1}{6} (39.2 + 39.2 + 39.2 + 39.1 + 39.2 + 39.2)=39.2
$$
$$
\sigma_N=0.04 \quad \sigma_{\bar{N}}=0.017 \quad \Delta A=0.043 \quad \Delta B=0.0095 \quad U=\sqrt{\Delta A^2+\Delta B^2}=0.044
$$
\quad 求得砝码质量$m$的平均测量值为
\textcolor{green!70!black}{
$$
m=(m \pm U)=(39.2\pm 0.0) \times 10^{-3}kg
$$}

\textcolor{blue!50!white}{圆环样品质量\quad$M_{\text{环}}$:[426.5,426.5,426.5,426.5,426.5,426.5]\quad 单位:$\times 10^{-3}kg$}
$$
\bar{N}=\frac{1}{6} (426.5 + 426.5 + 426.5 + 426.5 + 426.5 + 426.5)=462.5
$$
$$
\sigma_N=0.0 \quad \sigma_{\bar{N}}=0.0 \quad \Delta A=0.0 \quad \Delta B=0.0095 \quad U=\sqrt{\Delta A^2+\Delta B^2}=0.0095
$$
\quad 求得圆环样品质量$M_{\text{环}}$的平均测量值为
\textcolor{green!70!black}{
$$
M_{\text{环}}=(M_{\text{环}} \pm U)=(462.5\pm 0.0) \times 10^{-3}kg
$$}

\textcolor{blue!50!white}{圆环外径\quad$D_{\text{外}}$:[239.42,239.40,239.42,239.40,239.42,239.40]\quad 单位:$\times 10^{-3}m$}
$$
\bar{N}=\frac{1}{6} (239.42 + 239.40 + 239.42 + 239.40 + 239.42 + 239.40)=239.41
$$
$$
\sigma_N=0.011 \quad \sigma_{\bar{N}}=0.004 \quad \Delta A=0.011 \quad \Delta B=0.0095 \quad U=\sqrt{\Delta A^2+\Delta B^2}=0.01
$$
\quad 求得圆环外径$D_{\text{外}}$的平均测量值为
\textcolor{green!70!black}{
$$
D_{\text{外}}=(D_{\text{外}} \pm U)=(239.41\pm 0.01) \times 10^{-3}m
$$}

\textcolor{blue!50!white}{圆环内径\quad$D_{\text{内}}$:[210.40,210.38,210.38,210.40,210.38,210.38]\quad 单位:$\times 10^{-3}m$}
$$
\bar{N}=\frac{1}{6} (210.40 + 210.38 + 210.38 + 210.40 + 210.38 + 210.38)=210.39
$$
$$
\sigma_N=0.010 \quad \sigma_{\bar{N}}=0.004 \quad \Delta A=0.011 \quad \Delta B=0.0095 \quad U=\sqrt{\Delta A^2+\Delta B^2}=0.01
$$
\quad 求得圆环内径$D_{\text{外}}$的平均测量值为
\textcolor{green!70!black}{
$$
D_{\text{内}}=(D_{\text{内}} \pm U)=(210.39\pm 0.01) \times 10^{-3}m
$$}





\textcolor{blue!50!white}{圆柱1质量\quad$M_{\text{柱1}}$:[165.4,165.4,165.4,165.4,165.4,165.4]\quad 单位:$\times 10^{-3}kg$}
$$
\bar{N}=\frac{1}{6} (165.4 + 165.4 + 165.4 + 165.4 + 165.4 + 165.4)=165.4
$$
$$
\sigma_N=0.0 \quad \sigma_{\bar{N}}=0.0 \quad \Delta A=0.0 \quad \Delta B=0.0095 \quad U=\sqrt{\Delta A^2+\Delta B^2}=0.0
$$
\quad 求得圆柱1质量$M_{\text{柱1}}$的平均测量值为
\textcolor{green!70!black}{
$$
M_{\text{柱1}}=(M_{\text{柱1}} \pm U)=(165.4\pm 0.0) \times 10^{-3}kg
$$}

\textcolor{blue!50!white}{圆柱2质量\quad$M_{\text{柱2}}$:[164.8,164.7,164.7,164.7,164.8,164.8]\quad 单位:$\times 10^{-3}kg$}
$$
\bar{N}=\frac{1}{6} (164.8 + 164.7 + 164.7 + 164.7 + 164.8 + 164.8)=164.8
$$
$$
\sigma_N=0.055 \quad \sigma_{\bar{N}}=0.022 \quad \Delta A=0.057 \quad \Delta B=0.0095 \quad U=\sqrt{\Delta A^2+\Delta B^2}=0.01
$$
\quad 求得圆柱2质量$M_{\text{柱2}}$的平均测量值为
\textcolor{green!70!black}{
$$
M_{\text{柱2}}=(M_{\text{柱2}} \pm U)=(164.8\pm 0.1) \times 10^{-3}kg
$$}

\textcolor{blue!50!white}{圆柱1直径\quad$D_{\text{柱1}}$:[30.02,30.02,30.00,30.02,30.00,30.00]\quad 单位:$\times 10^{-3}m$}
$$
\bar{N}=\frac{1}{6} (30.02 + 30.02 + 30.00 + 30.02 + 30.00 + 30.00)=30.01
$$
$$
\sigma_N=0.010 \quad \sigma_{\bar{N}}=0.004 \quad \Delta A=0.011 \quad \Delta B=0.0095 \quad U=\sqrt{\Delta A^2+\Delta B^2}=0.01
$$
\quad 求得圆柱1直径$D_{\text{柱1}}$的平均测量值为
\textcolor{green!70!black}{
$$
D_{\text{柱1}}=(D_{\text{柱1}} \pm U)=(30.01\pm 0.01) \times 10^{-3}m
$$}

\textcolor{blue!50!white}{圆柱2直径\quad$D_{\text{柱2}}$:[30.00,30.00,29.98,30.00,29.98,30.00]\quad 单位:$\times 10^{-3}m$}
$$
\bar{N}=\frac{1}{6} (30.00 + 30.00 + 29.98 + 30.00 + 29.98 + 30.00)=29.99
$$
$$
\sigma_N=0.010 \quad \sigma_{\bar{N}}=0.004 \quad \Delta A=0.011 \quad \Delta B=0.0095 \quad U=\sqrt{\Delta A^2+\Delta B^2}=0.01
$$
\quad 求得圆柱2径$D_{\text{柱2}}$的平均测量值为
\textcolor{green!70!black}{
$$
D_{\text{柱2}}=(D_{\text{柱2}} \pm U)=(29.99\pm 0.01) \times 10^{-3}m
$$}


%\textcolor{green!70!black}{①通过添加砝码作出的F-Δy图像如下}\\


\large \textbf{\textcolor{blue}{Part2:β的测量}}\\

\textcolor{blue!50!white}{角加速度\quad$\beta_{1}$:[-0.0529, -0.0694, -0.0597, -0.0657, -0.0604, -0.0632, -0.0589, -0.0611, -0.0581]\quad 单位:$rad/s^2$}
$$
\bar{N}=\frac{1}{9}  (-0.0529 + -0.0694 + -0.0597 + -0.0657 + -0.0604 + -0.0632 + -0.0589 + -0.0611 + -0.0581)=-0.0610
$$
$$
\sigma_N=0.0047 \quad \sigma_{\bar{N}}=0.0016 \quad \Delta A=0.004 \quad \Delta B=0.0095 \quad U=\sqrt{\Delta A^2+\Delta B^2}=0.0103
$$
\quad 求得角加速度$\beta_{1}$的平均测量值为
\textcolor{green!70!black}{
$$
\beta_{1}=(\beta \pm U)=(-0.0610\pm 0.0103) rad/s^2
$$}

\textcolor{blue!50!white}{角加速度\quad$\beta_{2}$:[1.6854, 1.6628, 1.6745, 1.6565, 1.6635, 1.6448, 1.6544, 1.4753, 1.2807]\quad 单位:$rad/s^2$}
$$
\bar{N}=\frac{1}{9}  (1.6854 + 1.6628 + 1.6745 + 1.6565 + 1.6635 + 1.6448 + 1.6544 + 1.4753 + 1.2807)=1.5998
$$
$$
\sigma_N=0.1353 \quad \sigma_{\bar{N}}=0.0451 \quad \Delta A=0.1159 \quad \Delta B=0.0095 \quad U=\sqrt{\Delta A^2+\Delta B^2}=0.1163
$$
\quad 求得角加速度$\beta_{2}$的平均测量值为
\textcolor{green!70!black}{
$$
\beta_{2}=(\beta \pm U)=(1.5998\pm 0.1163) rad/s^2
$$}

\textcolor{blue!50!white}{角加速度\quad$\beta_{3}$:[-0.0503, -0.0330, -0.0405, -0.0335, -0.0378, -0.0337, -0.0364, -0.0336, -0.0357]\quad 单位:$rad/s^2$}
$$
\bar{N}=\frac{1}{9} (-0.0503 + -0.0330 + -0.0405 + -0.0335 + -0.0378 + -0.0337 + -0.0364 + -0.0336 + -0.0357)=-0.0372
$$
$$
\sigma_N=0.0055 \quad \sigma_{\bar{N}}=0.0018 \quad \Delta A=0.005 \quad \Delta B=0.0095 \quad U=\sqrt{\Delta A^2+\Delta B^2}=0.0106
$$
\quad 求得角加速度$\beta_{3}$的平均测量值为
\textcolor{green!70!black}{
$$
\beta_{3}=(\beta \pm U)=(-0.0372\pm 0.0106) rad/s^2
$$}





\textcolor{blue!50!white}{角加速度\quad$\beta_{4}$:[0.9013, 0.8893, 0.8956, 0.8853, 0.8920, 0.8816, 0.8844, 0.7790, 0.6756]\quad 单位:$rad/s^2$}
$$
\bar{N}=\frac{1}{9} (0.9013 + 0.8893 + 0.8956 + 0.8853 + 0.8920 + 0.8816 + 0.8844 + 0.7790 + 0.6756)=0.8538
$$
$$
\sigma_N=0.0765 \quad \sigma_{\bar{N}}=0.0255 \quad \Delta A=0.066 \quad \Delta B=0.0095 \quad U=\sqrt{\Delta A^2+\Delta B^2}=0.0662
$$
\quad 求得角加速度$\beta_{4}$的平均测量值为
\textcolor{green!70!black}{
$$
\beta_{4}=(\beta \pm U)=(0.8538\pm 0.0103) rad/s^2
$$}

\textcolor{blue!50!white}{角加速度\quad$\beta_{5}$:[-0.0312, -0.0369, -0.0325, -0.0345, -0.0322, -0.0344, -0.0336, -0.0340, -0.0324]\quad 单位:$rad/s^2$}
$$
\bar{N}=\frac{1}{9}  (-0.0312 + -0.0369 + -0.0325 + -0.0345 + -0.0322 + -0.0344 + -0.0336 + -0.0340 + -0.0324)=-0.0335
$$
$$
\sigma_N=0.0017 \quad \sigma_{\bar{N}}=0.0006 \quad \Delta A=0.014 \quad \Delta B=0.0095 \quad U=\sqrt{\Delta A^2+\Delta B^2}=0.0096
$$
\quad 求得角加速度$\beta_{5}$的平均测量值为
\textcolor{green!70!black}{
$$
\beta_{5}=(\beta \pm U)=(-0.0335\pm 0.0096) rad/s^2
$$}

\textcolor{blue!50!white}{角加速度\quad$\beta_{6}$:[1.0379, 1.0482, 1.0332, 1.0420, 1.0301, 1.0379, 1.0275, 1.0275, 0.9108]\quad 单位:$rad/s^2$}
$$
\bar{N}=\frac{1}{9} (1.0379 + 1.0482 + 1.0332 + 1.0420 + 1.0301 + 1.0379 + 1.0275 + 1.0275 + 0.9108)=1.0217
$$
$$
\sigma_N=0.0042 \quad \sigma_{\bar{N}}=0.0140 \quad \Delta A=0.0361 \quad \Delta B=0.0095 \quad U=\sqrt{\Delta A^2+\Delta B^2}=0.0373
$$
\quad 求得角加速度$\beta_{6}$的平均测量值为
\textcolor{green!70!black}{
$$
\beta_{6}=(\beta \pm U)=(1.0217\pm 0.0373) rad/s^2
$$}














\large \textbf{\textcolor{blue}{Part3:测量并计算放上圆环后的转动惯量并于理论值比较}}\\
\textcolor{blue!50!white}{
放上圆环后的转动惯量的实验值,由计算公式
}
\begin{equation*}
    J_1 = \frac{Rm(g - R\beta_2)}{\beta_2 - \beta_1}
\end{equation*}
\begin{equation*}
    J_2 = \frac{Rm(g - R\beta_4)}{\beta_4 - \beta_3}
\end{equation*}
\begin{equation*}
    J_3 = J_2 - J_1
\end{equation*}
可得

\textcolor{green!60!black}{
$$
J_1=0.0069 \quad J_2=0.0129\quad J_3=0.0060
$$}

\textcolor{blue!50!white}{
放上圆环后的理论计算值,由公式
}
\begin{equation*}
    J=\frac{m}{2}(R_{\text{外}}^2+R_{\text{内}}^2)
\end{equation*}
可得
\textcolor{green!60!black}{
$$
J=0.0054
$$}

相对误差 \textcolor{green!30!black}{
$$
E=\frac{J_3-J}{J}\times 100\%=10\%
$$}





\large \textbf{\textcolor{blue}{Part4:验证平行轴定理}}\\
\textcolor{blue!50!white}{
两圆柱体的转动惯量在与中心距离为d的实验值,由计算公式
}
\begin{equation*}
    J_1 = \frac{Rm(g - R\beta_2)}{\beta_2 - \beta_1}
\end{equation*}
\begin{equation*}
    J_2 = \frac{Rm(g - R\beta_6)}{\beta_6 - \beta_5}
\end{equation*}
\begin{equation*}
    J_3 = J_2 - J_1
\end{equation*}
可得

\textcolor{green!60!black}{
$$
J_1=0.0069 \quad J_2=0.0109\quad J_3=0.0040
$$}

\textcolor{blue!50!white}{
依据平行轴定理得出转动惯量的理论计算值,由公式
}
\begin{equation*}
    J_0=\frac{m_1}{2}R_1^2+\frac{m_2}{2}R_2^2=\frac{1}{2}\times 0.1654 \times (0.0150)^2+\frac{1}{2}\times 0.1648 \times (0.0150)^2=3.7147\times 10^{-5}
\end{equation*}
\begin{equation*}
(m_1+m_2)d^2=(0.1654+0.1648)\times 0.1125=0.0042
\end{equation*}
即
\textcolor{green!60!black}{
$$
J=J_0+md^2=0.0042
$$}

相对误差 \textcolor{green!30!black}{
$$
E=\frac{J_3-J}{J}\times 100\%=4.8\%
$$}


\large \textbf{\textcolor{blue}{Part5:思考与讨论}}\\
1.游标卡尺的一对内外测量爪之间均有缝隙,测量圆柱半径时注意减去相应距离\\
2.使用电子天平称重时应注意仪器的精度范围,从而确保实验数据的准确性\\
3.选择的悬挂砝码的细线应足够长以确保能收集到足够多的数据\\
4.实验过程应确保仪器水平放置,以减小重力等其它无关因素对实验的影响
































\end{enumerate}

\end{homeworkProblem}

%==========================================================换页==============================
\clearpage
\begin{homeworkProblem}[6]

\begin{enumerate}
	% alignment
	\item

\end{enumerate}

\end{homeworkProblem}












\end{document}
















































